\documentclass[UTF8,openright]{ctexbook}

% 论文版面要求:
% 统一按 word 格式A4纸(页面设置按word默认值)编排、打印、制作。
% 正文内容字体为宋体;字号为小4号;字符间距为标准;行距为25磅(约0.88175cm)。

%%%%% ===== 页面设置
\usepackage[a4paper,top=2.54cm,bottom=2.54cm,left=3.17cm,right=3.17cm,%
            ]{geometry}
            
\setlength{\parindent}{2em}
%默认的弹性间距会导致文中某些排版flush的时候,出现大量空白。
\setlength{\parskip}{0.5em} %指定固定段后间距,默认为弹性间距。
\setlength{\intextsep}{10pt} %固定浮浮动体前后间距。


%%%%% =====章节 标题 设置
\ctexset{%
  contentsname={\vspace{-3.5em}\centerline{\zihao{3}\heiti\textbf{ 目\quad 录}}\vspace{-0.7em}},
  listfigurename={\vspace{-3.5em}\centerline{\zihao{-3}\heiti 插\ 图\ 目\ 录}\vspace{-0.5em}},
  listtablename={\vspace{-3.5em}\centerline{\zihao{-3}\heiti 表\ 格\ 目\ 录}\vspace{-0.5em}},
  bibname={\vspace{-3em}\centerline{\zihao{-3}\heiti 参\ 考\ 文\ 献}\vspace{3em}},
  chapter={name={第,章},
           number=\chinese{chapter}, %指定章序号为一二三。。。。
           nameformat={\zihao{3}\bfseries},
           titleformat={\zihao{3}\bfseries},
           beforeskip={-10pt},
           afterskip={20pt}
           },
  section={format=\raggedright,
           nameformat={\zihao{4}\bfseries},
           titleformat={\zihao{4}\bfseries},
%           afterskip={1ex plus 0.2ex}
           beforeskip={1ex},% 固定段前段后间距,
           afterskip={1ex}
           },
  subsection={format=\raggedright,
           nameformat={\zihao{-4}\bfseries},
           titleformat={\zihao{-4}\bfseries},
%           afterskip={0.5ex plus 0.1ex}
           beforeskip={0.5ex},
           afterskip={0.5ex}
           }
}
%%%%% ===== 中英文字体
\setmainfont{Times New Roman}
\setCJKfamilyfont{STSong}{simsun.ttc}
\newcommand{\STSong}{\CJKfamily{STSong}}

%%%%% ===== 常用宏包
\usepackage{amsmath,amssymb,amsfonts,bm}
\usepackage[amsmath,thref,thmmarks,hyperref]{ntheorem}
\usepackage{graphicx,xcolor,float}
\usepackage{fancyhdr}
\usepackage{tocloft} % 设置目录中的条目间距


\renewcommand\cftdot{\textsubscript{……}}
\renewcommand\cftdotsep{0}

\setlength{\cftbeforechapskip}{1pt}
\renewcommand{\cftchapleader}{\cftdotfill{\cdot}}


\usepackage{booktabs} % toprule, midrule, bottomrule
\usepackage{varwidth} % 可变宽度的 parbox

%%%%% ===== 参考文献与链接
\usepackage[numbers,sort&compress,sectionbib,super, square]{natbib} %引用上标,禁用连续缩写。
\newcommand{\upcite}[1]{\textsuperscript{\cite{#1}}}


\usepackage[xetex,pagebackref]{hyperref}
\hypersetup{CJKbookmarks=true,colorlinks=true,citecolor=black,%
            linkcolor=black,urlcolor=black,bookmarksnumbered=true,%
	        bookmarksopen=true,breaklinks=true}

%%%%% ===== 算法
\usepackage{algorithm,algpseudocode}

%%%%% ===== 其他
\usepackage{ulem}
\def\ULthickness{1pt}




%%%%%===== Code Style代码
\usepackage{listings}
\usepackage{color}

\definecolor{dkgreen}{rgb}{0,0.6,0}
\definecolor{gray}{rgb}{0.5,0.5,0.5}
\definecolor{mauve}{rgb}{0.58,0,0.82}

\lstset{frame=tb,
  language=Scala,
  aboveskip=3mm,
  belowskip=3mm,
  showstringspaces=false,
  columns=flexible,
  basicstyle={\small\ttfamily},
  numbers=none,
  numberstyle=\tiny\color{gray},
  keywordstyle=\color{blue},
  commentstyle=\color{dkgreen},
  stringstyle=\color{mauve},
  breaklines=true,
  breakatwhitespace=true,
  tabsize=3
}

\begin{document}
%%%%% ===== 设置页面

%%%%% ===== 设置数学公式
\numberwithin{equation}{chapter}
\allowdisplaybreaks


%%%%% ===== 设置页眉和页脚
\pagestyle{fancy}
\fancyhf{}  % 清除以前对页眉页脚的设置

\newcommand{\makeheadrule}{%% 定义页眉与正文间双隔线
  \makebox[0pt][l]{\rule[.7\baselineskip]{\headwidth}{0.3pt}}%0.4
  \rule[0.85\baselineskip]{\headwidth}{1.0pt}\vskip-.8\baselineskip}
\makeatletter
\renewcommand{\headrule}{%
  {\if@fancyplain\let\headrulewidth\plainheadrulewidth\fi\makeheadrule}}
\makeatother
\renewcommand{\chaptermark}[1]{\markboth{\CTEXthechapter \ #1}{}}
\renewcommand{\sectionmark}[1]{\markright{\thesection \ #1}{}}
%\fancyhead[RO,LE]{{\small\songti\rightmark}}     % 节标题
%\fancyhead[RE]{{\small\songti\leftmark}}      % 章标题
\fancyhead[CO,CE]{华东师范大学硕士专业学位论文}
%\fancyhead[RO,LE]{$\cdot$ {\small\thepage} $\cdot$}
\fancyfoot[RO,LE]{{\thepage}}
%\fancyfoot[CO,CE]{{\thepage}}



%%%%% ===== 定理类环境
\theorempostskipamount0em
\theoremstyle{plain}
\theoremheaderfont{\normalfont\heiti\color{blue}}
\theorembodyfont{\normalfont\kaishu\color{black}}
\theoremindent0em
\theoremseparator{\hspace{0.2em}}
\theoremnumbering{arabic}
\newtheorem{theorem}{定理}[chapter]
\newtheorem{lemma}[theorem]{引理}
\newtheorem{corollary}[theorem]{推论}
\newtheorem{proposition}[theorem]{命题}
\newtheorem{property}[theorem]{性质}
\newtheorem{definition}{定义}[chapter]
\newtheorem{remark}{注记}[chapter]
%
%\theoremheaderfont{\normalfont\itshape\color{blue}}
\theorembodyfont{\normalfont\rmfamily\color{black}}
\newtheorem{example}{例}[chapter]

\theoremstyle{nonumberplain}
\theorempreskip{0em}
\theoremsymbol{\ensuremath{\Box}}
\newtheorem{proof}{证明}

%%%%% ===== 算法
\floatname{algorithm}{\color{blue} 伪代码}
\algrenewcommand{\algorithmiccomment}[1]{\quad{\color{red}\%\ #1}}
\numberwithin{algorithm}{chapter}
\renewcommand{\listalgorithmname}{算\ 法\ 目\ 录}


\def\tableautorefname{表}%
\def\figureautorefname{图}%

\def\equationautorefname{式}%
%\def\footnoteautorefname{脚注}%
%\def\itemautorefname{项}%
%\def\figureautorefname{图}%
%\def\tableautorefname{表}%
%\def\partautorefname{篇}%
%\def\appendixautorefname{附录}%
\def\chapterautorefname{章}%
\def\sectionautorefname{节}%
\def\subsectionautorefname{小节}%
%\def\subsubsectionautorefname{小小节}%
%\def\paragraphautorefname{段落}%
%\def\subparagraphautorefname{子段落}%
%\def\FancyVerbLineautorefname{行}%
%\def\theoremautorefname{定理}%
\def\algorithmautorefname{伪代码}

%%%%%%%%%%%%%%%%%%%%%%%%%%%%%%%%%%%%%%%%%%%%%%%%%%%%%%%%%%%%%%%%%%%%%%%
%%%%% ===== 封面 =====
\makeatletter

\newcommand{\mcc}[1]{\multicolumn{1}{c}{\underline{\makebox[10em][c]{#1}}}}
\newcommand{\mce}[1]{\multicolumn{1}{c}{\underline{\makebox[15em][l]{#1}}}}

\def\makecover{
\begin{titlepage}

%%%%% ===== 中文封面
  \pdfbookmark[0]{中文封面}{ccover}
  \linespread{1.1}\zihao{4}\ziju{0.05}
  \noindent{\@graduateyear\,届研究生硕士学位论文}\smallskip
  \par\noindent\hspace*{-12pt}
  \setlength{\tabcolsep}{2pt}
  \begin{tabular}[t]{rl}
    分\hspace{7.5pt}类\hspace{7.5pt}号:
                      &\ \underline{\makebox[6em][l]{\,\@class}}  \\[1.5ex]
    密\hspace{2em}级:&\ \underline{\makebox[6em][l]{\@security}}
  \end{tabular}
  \hfill
  \begin{tabular}[t]{rl}
    学校代码:&\ \underline{\makebox[6em][l]{\,10269}} \\[1.5ex]
    学\hspace{2em}号:&\ \underline{\makebox[6em][l]{\,\@studentid}}
  \end{tabular}

  \vspace{4em}
  \begin{center}
    \raisebox{1ex}[0pt][0pt]{
    \includegraphics[width=0.15\textwidth]{ecnu_logo}}\ \
    \includegraphics[width=0.75\textwidth,height=5em]{ecnu}

    \bigskip
    {\STSong
    % {\songti
      \zihao{-2}\textbf{East China Normal University} \\[0.5ex]
      \zihao{2} 硕士学位论文\\[1ex]
      \zihao{-2} \textbf{MASTER'S  DISSERTATION}

    % title
    \vspace{3em}\noindent
    \parbox[t]{0.25\textwidth}{\zihao{2} 论文题目:}
    \begin{varwidth}[t]{.75\linewidth}\linespread{1.3}\zihao{1}\textbf{\uline{\@ctitle}} \linebreak\end{varwidth}


    \vfill\linespread{1.5}\selectfont\zihao{4}
    \renewcommand{\arraystretch}{1.2}
    \begin{tabular}{p{0cm}p{5.5em}@{\extracolsep{0.5ex}}cc}
     ~ & 院 \hfill 系: & & \mcc{\@caffil } \\
     ~ & 专 \hfill 业: & & \mcc{\@cmajor}\\
     ~ & 研 \hfill 究 \hfill 方 \hfill 向:& & \mcc{\@cdirection}\\
     ~ & 指\hfill 导\hfill 老\hfill 师:& & \mcc{\textbf{\@csupervisor}}\\
     ~ & 学\hfill 位\hfill 申\hfill 请\hfill 人:& & \mcc{\textbf{\@cauthor}}\\
    \end{tabular}
    }

    \vfill\@cdate
  \end{center}

\clearpage % {\pagestyle{empty}\cleardoublepage}
%%%%% ===== 英文封面
  \thispagestyle{empty}
  \pdfbookmark[0]{英文封面}{ecover}
  \noindent Dissertation for master degree in \@graduateyear
  \hfill University Code:\, 10269\par\medskip
  \mbox{}\hfill Student ID:\, \@studentid


  \vspace{5em}
  \begin{center}
    {\zihao{-0} East China Normal University}
    %\scalebox{1.15}[1.6]{\bfseries EAST CHINA NORMAL UNIVERSITY}

    % title
    \vspace{6em}{\bfseries
    \parbox[t]{0.125\textwidth}{\zihao{2} Title:}
    \begin{varwidth}[t]{.75\linewidth}\linespread{1.4}\zihao{2} \uline{\@etitle} \linebreak\end{varwidth}}

    \vfill\linespread{1.5}\selectfont\mdseries

    \renewcommand{\arraystretch}{1.1}
    \scalebox{1}[1.0]{\setlength{\tabcolsep}{0.5ex}
    \begin{tabular}{rl}
     Department: &  \mce{\@eaffil } \\
     Major:      &  \mce{\@emajor}\\
     Research Direction: &  \mce{\@edirection}\\
     Supervisor: &  \mce{\@esupervisor}\\
     Candidate:  & \mce{\@eauthor}\\
    \end{tabular}
    }

    \vfill\@edate
  \end{center}

\end{titlepage}
\makeatother
}



\def\thss@int@infoitema#1{
	\@namedef{#1}##1{\@namedef{@#1}{##1}}
	\define@key{thss@info}{#1}{\@nameuse{#1}{##1}}
}
% eg. `\thss@int@infoitemb{cuniversity}' will expand to:
%   \define@key{thss@info}{cuniversity}{\def\cuniversity{#1}}
\def\thss@int@infoitemb#1{
	\define@key{thss@info}{#1}{\@namedef{#1}{##1}}
}
% Set up document information entries.
\thss@int@infoitema{graduateyear}
\thss@int@infoitema{class}
\thss@int@infoitema{security}
\thss@int@infoitema{ctitle}
\thss@int@infoitema{caffil}
\thss@int@infoitema{cmajor}
\thss@int@infoitema{cdirection}
\thss@int@infoitema{csupervisor}
\thss@int@infoitema{cauthor}
\thss@int@infoitema{studentid}
\thss@int@infoitema{cdate}
\thss@int@infoitema{etitle}
\thss@int@infoitema{eaffil}
\thss@int@infoitema{emajor}
\thss@int@infoitema{edirection}
\thss@int@infoitema{esupervisor}
\thss@int@infoitema{eauthor}
\thss@int@infoitema{edate}

% Set up document information using the `key = value' grammar.
\newcommand*{\studentinfo}[1]{\setkeys{thss@info}{#1}}
\zihao{-4}
\studentinfo{
    graduateyear = {2020},
	class={ },
	security={ },
	ctitle={标题标题标题标题标题标题\linebreak 标题标题标题标题标题标题},
	caffil={计算机科学与技术学院},
	cmajor={计算机技术},
	cdirection={嵌入式系统}, % 数值代数
	csupervisor={xxx\, 副教授},
	cauthor={陈xx},
	studentid={51174506xxx},
	cdate={20xx 年 10 月},
	etitle={Research and Implementation of  \linebreak What You want to do},
    eaffil={School of Computer Science and Technology} ,
    emajor={Computer Technology}, % Computational Mathematics
    edirection={Embedded System}, % Numerical Algebra
    esupervisor={Prof. XXX Shen},
    eauthor={XXX Chen},
    edate={May, 2021}
}
\def\cctitle{标题标题标题标题标题标题标题标题标题标题标题标题}
\def\ccauthor{\@cauthor}

%%%%% ===== 生成封面 ===== %使用不同的页边距
\newgeometry{top=2.0cm,bottom=2.0cm,left=2.5cm,right=2.5cm}
{
\renewcommand{\baselinestretch}{1.6}
\makecover
}

%%%%% ===== 原创性声明与著作权使用声明 =====
\include{Declaration}

%%%%% ===== 答辩委员会成员 =====
\include{Committee}

\frontmatter

\makeatletter
\let\ps@plain\ps@fancy% Let 'plain' be exactly the same as 'fancy'
\makeatother
\pagestyle{fancy} %使用fancy页眉和页脚
\fancyfoot[LO,LE]{}
\fancyfoot[RO,RE]{{\thepage}} %单页打印,此处页码固定右下角
%%%%% ===== 中文摘要 =====
\include{Abstract_chs}
%%%%% ===== 英文摘要 =====
\include{Abstract_eng}
%%%%% ===== 生成目录
\let\cleardoublepage\clearpage
%\pagestyle{empty}
\setcounter{tocdepth}{2} %向下显示到两级标题
\phantomsection % 修复引用问题
\addcontentsline{toc}{chapter}{目录}
\tableofcontents
%%%%%% ===== 正文部分 ===== %%%%%
\mainmatter
\linespread{1.6}\selectfont % 行距固定25磅,宋体
%\setlength{\baselineskip}{0.88175cm}
\pagestyle{fancy}
\fancyfoot[RO,RE]{}
\fancyfoot[LO,LE]{} %清空之前的页码排版。
\fancyfoot[RO,LE]{{\thepage}}%双面打印,页码向外排布。

\chapter{引言}
\section{研究背景和意义}


引言部分\cite{bi:hw-ml-challenges}, 介绍论文研究课题的应用背景或者问题来源,\cite{bi:hw-sw-co-design}
一些基本概念, 现有成果等等.


\section{国内外研究现状}


现有成果介绍, 现有成果介绍, 现有成果介绍, 现有成果介绍, 现有成果介绍.
现有成果介绍, 现有成果介绍, 现有成果介绍, 现有成果介绍, 现有成果介绍.
现有成果介绍, 现有成果介绍, 现有成果介绍, 现有成果介绍, 现有成果介绍.
现有成果介绍, 现有成果介绍, 现有成果介绍, 现有成果介绍, 现有成果介绍.
现有成果介绍, 现有成果介绍, 现有成果介绍, 现有成果介绍, 现有成果介绍.
现有成果介绍, 现有成果介绍, 现有成果介绍, 现有成果介绍, 现有成果介绍.

\section {本研究工作内容和章节安排}


\chapter {相关技术原理的分析与研究}


\section {xxx相关技术原理分析与研究}


\section {本章小节}


这是小节,这是小节,这是小节,这是小节,这是小节,这是小节,
这是小节,这是小节,这是小节,这是小节,这是小节,这是小节,
这是小节,这是小节,这是小节,这是小节,这是小节,这是小节,
这是小节,这是小节,这是小节,这是小节,这是小节,这是小节,

这是小节,这是小节,这是小节,这是小节,这是小节,这是小节,
这是小节,这是小节,这是小节,这是小节,这是小节,这是小节,
这是小节,这是小节,这是小节,这是小节,这是小节,这是小节,
这是小节,这是小节,这是小节,这是小节,这是小节,这是小节,

\clearpage %{\pagestyle{empty}\cleardoublepage}

\chapter {相关技术原理的分析与研究}


\section {xxx相关技术原理分析与研究}


\section {本章小节}


这是小节,这是小节,这是小节,这是小节,这是小节,这是小节,
这是小节,这是小节,这是小节,这是小节,这是小节,这是小节,
这是小节,这是小节,这是小节,这是小节,这是小节,这是小节,
这是小节,这是小节,这是小节,这是小节,这是小节,这是小节,

这是小节,这是小节,这是小节,这是小节,这是小节,这是小节,
这是小节,这是小节,这是小节,这是小节,这是小节,这是小节,
这是小节,这是小节,这是小节,这是小节,这是小节,这是小节,
这是小节,这是小节,这是小节,这是小节,这是小节,这是小节,

\clearpage %{\pagestyle{empty}\cleardoublepage}

\chapter {相关算法的研究}


\section {相关技术原理简述}


\subsection {xxx算法简述}\label{section:acoustic-feat}

\subsection {xxx模型简述}



\begin{equation}
\label{eq:DFT}
A_k = \sum_{n=0}^{N-1} {e^{-i {2 \pi \over N}}}^{kn} a_n 
\end{equation}


这是对公式的引用,\autoref{eq:DFT}中的引用

\section{本章小节}

本章我们对设计的xxx架构进行了详细的分析与实际性能对比测试。
%%%% ===== 参考文献 =====
\setlength{\bibsep}{1ex}  % 需 natbib 宏包
\begin{thebibliography}{99}
\addcontentsline{toc}{chapter}{参考文献}
\thispagestyle{fancy}

% \addtolength{\itemsep}{-5pt}


\bibitem{bi:hw-ml-challenges}
Sze V, Chen Y H, Emer J, et al. Hardware for machine learning: Challenges and opportunities[C]//2017 IEEE Custom Integrated Circuits Conference (CICC). IEEE, 2017: 1-8.

\bibitem{bi:hw-sw-co-design}
Brian Bailey, Hardware-Software Co-Design Reappears. semiengineering .2019-07-25. https://semiengineering.com/hardware-software-co-design-reappears/
 



\end{thebibliography}
% \chapter{引言}
% \section{研究背景和意义}


% 引言部分\cite{bi:hw-ml-challenges}, 介绍论文研究课题的应用背景或者问题来源,\cite{bi:hw-sw-co-design}
% 一些基本概念, 现有成果等等.


% \section{国内外研究现状}


% 现有成果介绍, 现有成果介绍, 现有成果介绍, 现有成果介绍, 现有成果介绍.
% 现有成果介绍, 现有成果介绍, 现有成果介绍, 现有成果介绍, 现有成果介绍.
% 现有成果介绍, 现有成果介绍, 现有成果介绍, 现有成果介绍, 现有成果介绍.
% 现有成果介绍, 现有成果介绍, 现有成果介绍, 现有成果介绍, 现有成果介绍.
% 现有成果介绍, 现有成果介绍, 现有成果介绍, 现有成果介绍, 现有成果介绍.
% 现有成果介绍, 现有成果介绍, 现有成果介绍, 现有成果介绍, 现有成果介绍.

% \section {本研究工作内容和章节安排}



\end{document}}